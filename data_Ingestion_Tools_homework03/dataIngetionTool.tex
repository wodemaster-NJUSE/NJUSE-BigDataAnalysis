\documentclass{article}

\usepackage{amsmath} % 用于数学公式
\usepackage{graphicx} % 用于插入图片
\usepackage{lipsum} % 用于生成虚拟文本
\usepackage{ctex} % 导入 ctex 包以支持中文
\usepackage{titlesec} % 导入 titlesec 包以定制标题样式
\usepackage{fontspec} % 用于设置中文字体

\setmainfont{SimSun} % 设置中文字体,SimSun 为宋体的系统字体

\usepackage{graphicx}
\usepackage{listings}
\usepackage{color}
\usepackage{float}

\title{数据流工具实验报告}
\author{程智镝}
\date{\today}

\begin{document}

\maketitle

\section*{涵盖工具}

本实验涵盖以下数据流工具:

\begin{enumerate}
    \item Apache Kafka
    \item AWS Kinesis
    \item Apache NiFi
    \item Flume
\end{enumerate}

\section*{实验任务}
\begin{enumerate}

    \item 任务一:使用 Apache Kafka 进行数据流
      \begin{itemize}
        \item 要求:安装 Apache Kafka。
        \item 任务:生产和消费一个简单的消息。
        \item 验证:确认 Kafka 主题中的消息。
      \end{itemize}
      
    \item 任务二:使用 AWS Kinesis 进行实时数据摄取
      \begin{itemize}
        \item 要求:在 AWS 控制台中创建一个 Kinesis 流。
        \item 任务:使用 AWS SDK 发送一批消息。
        \item 验证:在 Kinesis 控制台中监控传入数据。
      \end{itemize}
      
    \item 任务三:使用 Apache NiFi 进行数据流管理
      \begin{itemize}
        \item 要求:安装 Apache NiFi。
        \item 任务:创建一个简单的数据流,将数据从平面文件移动到数据库。
        \item 验证:确认数据库中的记录。
      \end{itemize}
      
    \item 任务四:使用 Flume 收集日志
      \begin{itemize}
        \item 要求:安装 Flume。
        \item 任务:配置 Flume 以收集日志并将其存储在 HDFS 中。
        \item 验证:确认 HDFS 中存储的日志。
      \end{itemize}

\end{enumerate}
\section*{实验难点}
\begin{itemize}
    \item 未使用过该配置
    \item 构造数据
    \item 我使用的是阿里云而不是AWS,配置上可能会有差距
\end{itemize}
\section*{任务一}
% \textbf(任务流程参照: \verb| https://blog.csdn.net/sun_hong_likeIT/article/details/123502688|)
% 下载apache kafka:sudo wget \verb|https://mirrors.tuna.tsinghua.edu.cn/apache/kafka/3.5.1/kafka_2.12-3.5.1.tgz|
% 这里使用的是清华镜像网站,比教程给的网站更好用。

% 使用 bin/zookeeper-server-start.sh -daemon config/zookeeper.properties 以守护进程启动)
% 启动kafa:bin/kafka-server-start.sh config/server.properties &
% 新建终端创建主题:bin/kafka-topics.sh --bootstrap-server localhost:9092 --create --topic test --partitions 2 --replication-factor 1

% 在主题终端发送消息:bin/kafka-console-producer.sh --broker-list localhost:9092 --topic test
% 服务终端接收到消息:bin/kafka-console-consumer.sh --bootstrap-server localhost:9092 --topic test --from-beginning

% 消息生产和消费过程:

\section*{任务二}
由于我使用的不是AWS,我在阿里云上使用了数据总线来进行平替:\verb|https://www.aliyun.com/product/bigdata/datahub|
\section*{任务三}
nifi安装:wget https://dlcdn.apache.org/nifi/1.23.2/nifi-1.23.2-source-release.zip\\
解压:unzip nifi-1.23.2-source-release.zip
用官网会很慢,建议使用国内镜像
\section*{任务四}
安装 Flume:https://downloads.apache.org/flume/1.11.0/apache-flume-1.11.0-bin.tar.gz
\end{document}
