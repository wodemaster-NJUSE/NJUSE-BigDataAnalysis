\documentclass{article}

\usepackage{amsmath} % 用于数学公式
\usepackage{graphicx} % 用于插入图片
\usepackage{lipsum} % 用于生成虚拟文本
\usepackage{ctex} % 导入 ctex 包以支持中文
\usepackage{titlesec} % 导入 titlesec 包以定制标题样式
\usepackage{fontspec} % 用于设置中文字体

\setmainfont{SimSun} % 设置中文字体,SimSun 为宋体的系统字体

\title{LinkPrediction实验}
\author{程智镝、陈凌}
\date{\today}

\begin{document}
\maketitle

\section*{作业任务}
\begin{enumerate}
    \item 从代码(自己实现or复现)、数据集(直接获取或自己处理得到)两个角度权衡是否选择某个link prediction的工作。。
    \item 论文摘要abstract和introduction翻译
    \item 问题描述。
    \item 输入、输出、模型算法描述(附框架图;有多个的挑1个主要实现)
    \item 评价指标及其计算公式
    \item 对比方法及这些对比方法的引用论文出处
    \item 结果
    \item 打包提交code、运行配置说明(数据集太大的可以是开放链接,需描述)
\end{enumerate}
\subsection*{实验难点:}
\begin{itemize}
    \item 论文为全英文描述,阅读难度提升
    \item 论文实验复现环境搭配
    \item 相关神经网络、机器学习、图论的知识暂且未知
\end{itemize}

\subsection*{论文选择:Sampling Enclosing Subgraphs for Link Prediction}
链接: \verb|https://arxiv.org/pdf/2206.12004.pdf|























\end{document}